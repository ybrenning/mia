\documentclass[12pt]{article}

\usepackage[utf8]{inputenc}
\usepackage{url}
\usepackage{latexsym,amsfonts,amssymb,amsthm,amsmath}
\usepackage{mathtools}

\setlength{\parindent}{0in}
\setlength{\oddsidemargin}{0in}
\setlength{\textwidth}{6.5in}
\setlength{\textheight}{8.8in}
\setlength{\topmargin}{0in}
\setlength{\headheight}{18pt}



\title{MIA Feuille d’exercices numéro 2}
\author{Yannick Brenning}

\begin{document}

\maketitle

\vspace{0.5in}



\subsection*{Exercice 2}
\begin{enumerate}
    \item 
    \begin{flalign*}
        & \text{On peut exprimer la probabilité d'une intersection d'évènements avec la formule} && \\
        & \text{des probabilités composées:} && \\
        & P(A_1 \cap \dots \cap A_n) = P(A_1) \cdot P(A_2 | A_1) \cdot \dots \cdot P(A_n | A_1 \cap \dots \cap A_{n-1}) && \\ && \\
        & \text{On utilisant la modèle graphique, on peut simplifier les évènements indépendants.} && \\
        & \text{Bien sûr que $TS$ est indépendant de tout les autres évènements, et $SlO, SuO$} && \\
        & \text{sont indépendants entre eux.} && \\ && \\
        & P(SlL, SuL, SlO, SuO, TS) && \\
        & = P(TS) \cdot P(SlO) \cdot P(SuO) \cdot P(SlL|SlO, TS) \cdot P(SuL|SuO, TS) && \\
    \end{flalign*}
    \item 
    Oui, parce que les valuers $ t, u $ et $ l $ contiennent implicitement l'information sur \\ $ P(SlL|SlO, TS) $ et $ P(SuL|SuO, TS) $. Cela ce voit dans les formules $ P(SlL = 1|SlO = a, TS = b) = a \lor b $ et $ P(SuL = 1|SuO = a, TS = b) = a \lor b $, qui sont equivalentes et directement liées à les valeurs de $ t, u $ et $ l $. \\ \\
    On peut donc exprimer la factorisation de la manière suivante: \\
    \begin{flalign*}
        P(SlL, SuL, SlO, SuO, TS) =  t \cdot u \cdot l
    \end{flalign*}
    \item 
    \begin{flalign*}
        & P(TS = 1|SlL = 1) = \frac{P(TS=1, SlL=1)}{P(SlL=1)} && \text{Définition de la prob. cond.} \\
        & = \frac{P(SlL=1|TS=1) \cdot P(TS=1)}{P(SlL=1)}
    \end{flalign*}
        Car $P(SlL=1|SlO=a, TS=b) = a \lor b,$ on voit que pour $ TS = 1$, la probabilité est toujours égale à 1, c'est-à-dire $P(SlL=1|TS=1)$ est l'événement certain. 
    \begin{flalign*}
        & = \frac{P(TS=1)}{P(SlO=1) + P(TS=1) - P(SlO=1, TS=1)} && \text{Définition de } P(A \text{ ou } B) \\
        & = \frac{P(TS=1)}{P(SlO=1) + P(TS=1) - P(SlO=1)\cdot P(TS=1)} && \\
        & = \frac{t}{l+t-l \cdot t}
    \end{flalign*}
    \item 
    \begin{flalign*}
        & P(SlO = 1|SlL = 1) = \frac{P(SlO=1, SlL=1)}{P(SlL=1)} && \\
        & \frac{P(SlL=1 | SlO=1) \cdot P(SlO=1)}{P(SlL=1)} && \\
        & = \frac{1 \cdot P(SlO=1)}{P(SlL=1)} && \\
        & = \frac{l}{l+t-l \cdot t}
    \end{flalign*}
    \item 
    \begin{flalign*}
        & P(TS = 1|SlL = 1, SuL = 1) = \frac{P(TS=1, SlL=1, SuL=1)}{P(SlL=1, SuL=1)} && \\
        & = \frac{P(TS=1) \cdot P(SlL=1|TS=1) \cdot P(SuL=1|TS=1)}{P(SlL=1, SuL=1)} && \\
        & = \frac{P(TS=1)}{P(SlL=1, SuL=1)}
    \end{flalign*}
    On peut calculer $P(SlL=1, SuL=1)$ en pensant à les opérateurs logiques. Si les deux sont en retard, soit $TS$, soit $SlO$ et $SuO$, ou bien les trois on eu lieu. 
    \begin{flalign*}
        & \frac{P(TS=1)}{P(SlL=1, SuL=1)} = \frac{t}{l\cdot u + t + l\cdot u \cdot t} &&
    \end{flalign*}
    \item 
    \begin{flalign*}
        & P(SlO=1|SlL=1,SuL=1) = \frac{P(SlO=1, SlL=1, SuL=1)}{P(SlL=1, SuL=1)} && \\
        & = \frac{P(SlO=1) \cdot P(SlL=1|SlO=1) \cdot P(SuL=1)}{P(SlL=1, SuL=1)} && \\
        & = \frac{l \cdot (u+t-u \cdot t)}{l\cdot u + t + l\cdot u \cdot t}
    \end{flalign*}
    \item 
    Pour répondre à cette question, il faut calculer $P(TS=1|SlL=1)$ et $P(SlO=1|SlL=1)$ et les comparer.
    \begin{flalign*}
        & P(TS=1|SlL=1) = \frac{P(TS=1, SlL=1)}{P(SlL=1)} && \\
        & = \frac{P(TS) \cdot P(SlL=1 | TS=1)}{P(SlL=1)} && \\
        & = \frac{P(TS)}{P(SlL=1)} = \frac{t}{l + t - l \cdot t} && \\
        & = \frac{0.1}{0.5+0.1-0.5 \cdot 0.1} \approx 0.18 && \\ && \\
        & P(SlO=1|SlL=1) = \frac{P(SlO=1, SlL=1)}{P(SlL=1)} && \\
        & = \frac{P(SlO=1) \cdot P(SlL=1 | SlO=1)}{P(SlL=1)} && \\
        & = \frac{P(SlO=1)}{P(SlL=1)} = \frac{l}{l+t-l \cdot t} && \\
        & = \frac{0.5}{0.5+0.1-0.5 \cdot 0.1} \approx 0.91
    \end{flalign*}
    Alors dans ce cas, ``Sleepy overslept'' est plus probable.
    \item 
    On fait les même calculations que ci-dessus avec $u = 0.01$ au lieu de $ l = 0.5$ pour obtenir $P(TS=1|SuL=1) \approx 0.92 $ et $P(SuO=1|SuL=1) \approx 0.09 $, ce qui nous dit que ``There is a train strike'' est plus probable.
    \item 
    La probabilité calculé dans la question précédante devient $P(SuO=1|SuL=1) \approx 0.71$.
\end{enumerate}

\subsection*{Exercice 6}

\begin{enumerate}
    \item 
    \begin{flalign*}
        & b(\hat{f}) = \mathbb{E}(\hat{f}) - f && \\
        & = \mathbb{E}(\frac{1}{n} \sum_{i=1}^n X_i) - \mu && \\
        & = [\frac{1}{n} \sum_{i=1}^n \mathbb{E}(X_i)] - \mu && \text{Linéarité de l'espérance} \\
        & = [\frac{1}{n} \cdot n \cdot \mu] - \mu && \text{$X_i$ ont les mêmes espérances}\\
        & = \mu - \mu = 0
    \end{flalign*}
    \begin{flalign*}
        & \text{Var}(\hat{f}) = \mathbb{E}[(\hat{f} - \mathbb{E}(\hat{f}))^2] && \\
        & = \mathbb{E}[(\hat{f} - \mu)^2] && \\
        & = \mathbb{E}[(\frac{1}{n}\sum_{i=1}^n X_i - \mu)^2] && \\
        & = \frac{1}{n^2}\sum_{i=1}^n \mathbb{E}[(X_i - \mu)^2]
    \end{flalign*}
    \item 
    \begin{flalign*}
        & b(\hat{V}) = \mathbb{E}[\frac{1}{n}\sum_{i=1}^n(X_i-\hat{\mu})^2] - \mathbb{E}[(\hat{f} - \mathbb{E}(\hat{f})^2] && \\
    \end{flalign*}
\end{enumerate}

\end{document}
