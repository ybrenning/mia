\documentclass[12pt]{article}

\usepackage[utf8]{inputenc}
\usepackage{url}
\usepackage{latexsym,amsfonts,amssymb,amsthm,amsmath}
\usepackage{mathtools}

\setlength{\parindent}{0in}
\setlength{\oddsidemargin}{0in}
\setlength{\textwidth}{6.5in}
\setlength{\textheight}{8.8in}
\setlength{\topmargin}{0in}
\setlength{\headheight}{18pt}



\title{MIA Feuille d’exercices numéro 2}
\author{Yannick Brenning}

\begin{document}

\maketitle

\vspace{0.5in}



\subsection*{Exercice 2}
\begin{enumerate}
    \item 
    \begin{flalign*}
        & \text{On peut exprimer la probabilité d'une intersection d'évènements avec la formule} && \\
        & \text{des probabilités composées:} && \\
        & P(A_1 \cap \dots \cap A_n) = P(A_1) \cdot P(A_2 | A_1) \cdot \dots \cdot P(A_n | A_1 \cap \dots \cap A_{n-1}) && \\ && \\
        & \text{On utilisant la modèle graphique, on peut simplifier les évènements indépendants.} && \\
        & \text{Bien sûr que $TS$ est indépendant de tout les autres évènements, et $SlO, SuO$} && \\
        & \text{sont indépendants entre eux.} && \\ && \\
        & P(SlL, SuL, SlO, SuO, TS) && \\
        & = P(TS) \cdot P(SlO) \cdot P(SuO) \cdot P(SlL|SlO, TS) \cdot P(SuL|SuO, TS) && \\
    \end{flalign*}
    \item 
    Oui, parce que les valuers $ t, u $ et $ l $ contiennent implicitement l'information sur \\ $ P(SlL|SlO, TS) $ et $ P(SuL|SuO, TS) $. Cela ce voit dans les formules $ P(SlL = 1|SlO = a, TS = b) = a \lor b $ et $ P(SuL = 1|SuO = a, TS = b) = a \lor b $, qui sont equivalentes et directement liées à les valeurs de $ t, u $ et $ l $. \\ \\
    On peut donc exprimer la factorisation de la manière suivante: \\
    \begin{flalign*}
        P(SlL, SuL, SlO, SuO, TS) =  t \cdot u \cdot l
    \end{flalign*}
    \item 
    \begin{flalign*}
        & P(TS = 1|SlL = 1) = \frac{P(TS=1, SlL=1)}{P(SlL=1)} && \\
        & = \frac{P(SlL=1|TS=1) \cdot P(TS=1)}{P(SlL=1)} = \frac{P(TS=1)}{P(SlO=1) + P(TS=1) - P(SlO=1, TS=1)} && \\
        & = \frac{P(TS=1)}{P(SlO=1) + P(TS=1) - P(SlO=1)P(TS=1)} = \frac{t}{l+t-l \cdot t}
    \end{flalign*}
    \item 
    \begin{flalign*}
        & P(SlO = 1|SlL = 1) = \frac{P(SlO, SlL)}{P(SlL)} && \\
        & \frac{P(SlO) \cdot P(SlL)}{P(SlL} = P(SlO) = l
    \end{flalign*}
    \item 
    \begin{flalign*}
        & P(TS = 1|SlL = 1, SuL = 1) = \frac{P(TS, SlL, SuL)}{P(SlL, SuL)} && \\
        & = \frac{P(TS) \cdot P(SlL|TS) \cdot P(SuL|TS)}{P(SlL) \cdot P(SuL)} = \frac{P(TS) \cdot }{}
    \end{flalign*}
\end{enumerate}


\end{document}
