\documentclass[12pt]{article}

\usepackage[utf8]{inputenc}
\usepackage{url}
\usepackage{latexsym,amsfonts,amssymb,amsthm,amsmath}
\usepackage{mathtools}

\setlength{\parindent}{0in}
\setlength{\oddsidemargin}{0in}
\setlength{\textwidth}{6.5in}
\setlength{\textheight}{8.8in}
\setlength{\topmargin}{0in}
\setlength{\headheight}{18pt}



\title{MIA Feuille d’exercices numéro 3}
\author{Yannick Brenning}

\begin{document}

\maketitle

\vspace{0.5in}



\subsection*{Exercice 5}
\begin{enumerate}
    \item 
    \begin{flalign*}
        & B = (3 e_1 - 2 e_2, e_1 + e_2) \Leftrightarrow ((3, -2), (1, 1)) && \\ && \\
        & \text{Il suffit de montrer que la famille de vecteurs $B$ est linéairement indépendant:} && \\ && \\
        & 3 a_1 + a_2 = 0 && \\
        & -2 a_1 + a_2 = 0 && \\
        & \Rightarrow -3a_1 = a_2 = 2a_1 && \\
        & \Rightarrow a_1 = a_2 = 0 && \\ && \\
        & \text{La seule combinaison linéaire des deux vecteurs égale au vecteur zéro est celle} && \\ 
        & \text{dont tous les coéfficients sont nuls. Les deux vecteurs sont alors linéairement} && \\
        & \text{indépendants et forment une base dans $\mathbb{R}^2$.}
    \end{flalign*}
    \item 
    \begin{flalign*}
        & v = v_1 \begin{pmatrix}
            1 \\
            0
        \end{pmatrix}
        +v_2 \begin{pmatrix}
            0 \\
            1
        \end{pmatrix}
        = \begin{pmatrix}
            v_1 \\
            0
        \end{pmatrix} +
        \begin{pmatrix}
            0 \\
            v_2
        \end{pmatrix} 
        = \begin{pmatrix}
            v_1 \\
            v_2
        \end{pmatrix} && \\
        & \begin{pmatrix}
            1 \\
            0
        \end{pmatrix}
        = a \begin{pmatrix}
            3 \\
            -2
        \end{pmatrix}
        + b \begin{pmatrix}
            1 \\
            1
        \end{pmatrix} && \\
        & \Leftrightarrow 1= 3a + b, 0 = -2a + b \Rightarrow 2a = b \Rightarrow a = 1/5, b = 2/5 && \\
        & \begin{pmatrix}
            0 \\
            1
        \end{pmatrix}
        = c \begin{pmatrix}
            3 \\
            -2
        \end{pmatrix}
        + d \begin{pmatrix}
            1 \\
            1
        \end{pmatrix}  && \\
        & \Leftrightarrow 0 = 3c + d, 1 = -2c + d \Rightarrow d = -3c \Rightarrow c = -1/5, d = 3/5 && \\ && \\
        & \text{Donc } P = \begin{pmatrix}
            1/5 & -1/5 \\
            2/5 & 3/5
        \end{pmatrix}
    \end{flalign*}
    \item 
    \begin{flalign*}
        \begin{pmatrix}
            1/5 & -1/5 \\
            2/5 & 3/5
        \end{pmatrix}
        \begin{pmatrix}
            -1 \\
            1
        \end{pmatrix} 
        = \begin{pmatrix}
            -2/5 \\
            1/5
        \end{pmatrix} && \\
    \end{flalign*}
\end{enumerate}

\subsection*{Exercice 7}

\begin{enumerate}
    \item 
    \begin{enumerate}
        \item 
        En utilisant la notation de séance 6, on peut écrire:
        \begin{flalign*}
            & C = AB = \sum_i a_i \Tilde{b}_i^T
        \end{flalign*}
        Avec $ a_i $ la $i$-ième colonne de la matrice de gauche et $ \Tilde{b}_i^T $ la transpose de la $i$-ième ligne de la matrice de droite. Dans ce cas, cela donne une combinaison linéaire des colonnes de la matrice de gauche avec les valeurs scalaires du vecteur de droite.
        \begin{flalign*}
            \begin{pmatrix}
                3 \\
                2
            \end{pmatrix} \cdot 4
            + \begin{pmatrix}
                6 \\
                1
            \end{pmatrix} \cdot 1
            + \begin{pmatrix}
                2 \\
                3
            \end{pmatrix} \cdot 1
            = \begin{pmatrix}
                20 \\
                12
            \end{pmatrix}
        \end{flalign*}
        \item 
        \begin{flalign*}
            C = AB = \begin{pmatrix}
                \Tilde{a}_1^T B \\
                \vdots \\
                \Tilde{a}_m^T B
            \end{pmatrix}
        \end{flalign*}
        Car la matrice de droite $B$ est un vecteur dans ce cas, le resultat de cette expression devient une matrice contenant deux produits scalaires. 
        \begin{flalign*}
            \begin{pmatrix}
                \begin{pmatrix}
                    3 & 6 & 2
                \end{pmatrix}^T
                \begin{pmatrix}
                    4 \\
                    1 \\
                    1
                \end{pmatrix} \\ \\
                \begin{pmatrix}
                    2 & 1 & 3
                \end{pmatrix}^T
                \begin{pmatrix}
                    4 \\
                    1 \\
                    1
                \end{pmatrix}
            \end{pmatrix}
            = \begin{pmatrix}
                3 \cdot 4 + 6 \cdot 1 + 2 \cdot 1 \\
                2 \cdot 4 + 1 \cdot 1 + 3 \cdot 1
            \end{pmatrix}
            = \begin{pmatrix}
                20 \\
                12
            \end{pmatrix}
        \end{flalign*}
    \end{enumerate}
    \item 
    \begin{enumerate}
        \item 
        \begin{align*}
            C & = AB = \begin{pmatrix}
                A b_1 & \dots & A b_p
            \end{pmatrix} && \\ && \\
            & = \begin{pmatrix}
                A 
                \begin{pmatrix}
                    4 \\
                    1 \\
                    1
                \end{pmatrix}
                & A
                \begin{pmatrix}
                    5 \\
                    2 \\
                    -1
                \end{pmatrix}
            \end{pmatrix}
            = \begin{pmatrix}
                20 & 25 \\
                12 & 9
            \end{pmatrix}
        \end{align*}
        \item 
        \begin{align*}
            C = AB = \begin{pmatrix}
                \Tilde{a}_1^T B \\
                \vdots \\
                \Tilde{a}_m^T B
            \end{pmatrix} && \\ && \\
            \begin{pmatrix}
                \begin{pmatrix}
                    3 & 6 & 2
                \end{pmatrix}^T
                B \\ \\
                \begin{pmatrix}
                    2 & 1 & 3
                \end{pmatrix}^T
                B
            \end{pmatrix}
            = \begin{pmatrix}
                20 & 25 \\
                12 & 9 
            \end{pmatrix}
        \end{align*}
        \item 
        \begin{align*}
            C = AB = a_1 \Tilde{b}_1
            + a_2 \Tilde{b}_2
            + a_3 \Tilde{b}_3 && \\ && \\
            \begin{pmatrix}
                12 & 15 \\
                8 & 10
            \end{pmatrix}
            + \begin{pmatrix}
                6 & 12 \\
                1 & 2
            \end{pmatrix}
            + \begin{pmatrix}
                2 & -2 \\
                3 & -3
            \end{pmatrix}
            = \begin{pmatrix}
                20 & 25 \\
                12 & 9
            \end{pmatrix}
        \end{align*}
        \item 
        \begin{align*}
            C = AB = 
            \begin{pmatrix}
                \langle \Tilde{a}_1^T, b_1 \rangle & \langle \Tilde{a}_1^T, b_2 \rangle \\
                \langle \Tilde{a}_2^T, b_1 \rangle & \langle \Tilde{a}_2^T, b_2 \rangle
            \end{pmatrix} \\ \\
            \begin{pmatrix}
                \begin{pmatrix}
                    3 \\
                    6 \\
                    2
                \end{pmatrix} \cdot
                \begin{pmatrix}
                    4 \\
                    1 \\
                    1
                \end{pmatrix} &
                \begin{pmatrix}
                    3 \\
                    6 \\
                    2
                \end{pmatrix} \cdot
                \begin{pmatrix}
                    5 \\
                    2 \\
                    -1
                \end{pmatrix} \\ \\
                \begin{pmatrix}
                    2 \\
                    1 \\
                    3
                \end{pmatrix} \cdot
                \begin{pmatrix}
                    4 \\
                    1 \\
                    1
                \end{pmatrix} &
                \begin{pmatrix}
                    2 \\
                    1 \\
                    3
                \end{pmatrix} \cdot
                \begin{pmatrix}
                    5 \\
                    2 \\
                    -1
                \end{pmatrix}
            \end{pmatrix}
            = \begin{pmatrix}
                20 & 25 \\
                12 & 9
            \end{pmatrix}
        \end{align*}
    \end{enumerate}
\end{enumerate}

\subsection*{Exercice 8}

\begin{enumerate}
    \item 
    On dénote les vecteurs colonne de $ A $ par $ a_i $, et $x_i$ sont les valeurs scalaires (les coefficients diagonaux). \\ 
    \begin{align*}
        A \text{diag}(x_1, \dotsc, x_n) & = 
        \begin{pmatrix}
            a_{11} & \dots & a_{1n} \\
            \vdots & \ddots & \vdots \\
            a_{p1} & \dots & a_{pn}
        \end{pmatrix}
        \begin{pmatrix}
            x_1 & & \\
             & \ddots & \\
             & & x_n
        \end{pmatrix} \\ \\
        & = \begin{pmatrix}
            a_1 & \dots & a_n
        \end{pmatrix}
        \begin{pmatrix}
            x_1 & & \\
             & \ddots & \\
             & & x_n
        \end{pmatrix} \\ \\
        & = \begin{pmatrix}
            a_1 x_1 & \dots & a_n x_n
        \end{pmatrix}
    \end{align*}
    \item 
    \begin{flalign*}
        \text{diag}(x_1, \dotsc, x_n) B & = 
        \begin{pmatrix}
            x_1 & & \\
             & \ddots & \\
             & & x_n
        \end{pmatrix} 
        \begin{pmatrix}
            b_{11} & \dots & b_{1p} \\
            \vdots & \ddots & \vdots \\
            b_{n1} & \dots & b_{np}
        \end{pmatrix} \\ \\
        & = \begin{pmatrix}
            x_1 & & \\
             & \ddots & \\
             & & x_n
        \end{pmatrix} 
        \begin{pmatrix}
            \Tilde{b}_1^T \\
            \vdots \\
            \Tilde{b}_n^T
        \end{pmatrix} \\ \\
        & = \begin{pmatrix}
            x_1 \Tilde{b}_1^T \\
            \vdots \\
            x_n \Tilde{b}_n^T
        \end{pmatrix}
    \end{flalign*}
\end{enumerate}

\subsection*{Exercice 15}

\begin{enumerate}
    \item 
    \begin{align*}
        AV & = U \Sigma && \\
        AVV^T & = U \Sigma V^T && \text{Multiplication à droite par $V^T$}\\
        \Leftrightarrow AI_n & = U \Sigma V^T && \text{Orthogonalité de $V$} \\
        \Leftrightarrow A & = U \Sigma V^T && \text{Multiplication par matrice identité}
    \end{align*}
    \begin{align*}
        AV & = U \Sigma && \\
        U^T A V & = U^T U \Sigma && \text{Multiplication à gauche par $U^T$} \\
        \Leftrightarrow I_m \Sigma &= U^T A V && \text{Orthogonalité de $U$} \\
        \Leftrightarrow \Sigma & = U^T A V && \text{Multiplication par matrice identité}
    \end{align*}
    \item 
    \begin{align*}
        AV &= U \Sigma && \\
        \Leftrightarrow \begin{pmatrix}
            Av_1 & \dots & Av_n 
        \end{pmatrix} & = U \Sigma && \text{Exercice 7.2a} \\
        \Leftrightarrow \begin{pmatrix}
            Av_1 & \dots & Av_n 
        \end{pmatrix} & = U \text{diag}(\sigma_1, \dotsc, \sigma_r, 0 \dotsc, 0) && \\
        \Leftrightarrow \begin{pmatrix}
            Av_1 & \dots & Av_n 
        \end{pmatrix} & = \begin{pmatrix}
            u_1 \sigma_1 & \dots & u_r \sigma_r & u_{r+1} \cdot 0 & \dots & u_n \cdot 0
        \end{pmatrix} && \text{Exercice 8.1} \\
        \Leftrightarrow \forall i \in \{1, \dotsc, r\}: Av_i & = u_i \sigma_i  
    \end{align*}
    \item 
    \begin{align*}
        \begin{pmatrix}
               Av_1 & \dots & Av_n 
        \end{pmatrix} & = \begin{pmatrix}
            u_1 \sigma_1 & \dots & u_r \sigma_r & u_{r+1} \cdot 0 & \dots & u_n \cdot 0
        \end{pmatrix} &&  \\
        \Leftrightarrow \begin{pmatrix}
            Av_{r+1} & \dots & Av_n
        \end{pmatrix}
        & = \begin{pmatrix}
            u_{r+1} \cdot 0 & \dots & u_{n} \cdot 0
        \end{pmatrix} && \\
        \Leftrightarrow \begin{pmatrix}
            Av_{r+1} & \dots & Av_n
        \end{pmatrix}
        & = \begin{pmatrix}
            0 & \dots & 0
        \end{pmatrix} && \\
        \Leftrightarrow \forall i \in \{r+1, \dotsc, n\}: Av_i & = 0
    \end{align*}
    \item 
    \begin{align*}
        AV = U\Sigma & \Leftrightarrow A = U \Sigma V^T && \\
        \Leftrightarrow A & = U \begin{pmatrix}
            \sigma_1 v_1^T & \dots & \sigma_r v_r^T
        \end{pmatrix} && \\
        \Leftrightarrow A & = \sum_{i = 1}^r u_i \sigma_i v_i^T && \text{Exercice 7.1a} \\
        \Leftrightarrow A & = \sum_{i = 1}^r \sigma_i u_i v_i^T && \text{Loi commutative pour les coéfficients réels $\sigma_i$}
    \end{align*}
\end{enumerate}
\end{document}
